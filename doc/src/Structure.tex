\section{Project Structure}

The BPMConverter is build with two purposes.
On the one hand it provides interfaces and classes to describe Process Models using different representations.
There are elements to create an so called Activity Centric Process model as well as elements to create an Object Life Cycle.
These classes can be found in the \textit{activity\_centric} package.

On the other hand their are classes implementing different algorithms to generate additional process models.
These classes can be found inside the \textit{converter} package.

\subsection{How to create a process model}

The \textit{conversion} package contains classes to build process models.
In general a model consists of multiple nodes connected by edges.
All classes representing a node must implement the \textit{INode} interface.
Edges must implement the \textit{IEdge} interface.
Each node can have multiple incoming and outgoing edges, but their might stricter restrictions for the concrete implementations.

Currently their are two types of models implemented:
\begin{itemize}
	\item \textbf{Activity centric process models}. These models can be compared to BPMN models.
	\item \textbf{Object Life Cycles}. Instances of this model represent a state transition chart for one data class.
\end{itemize}

\subsubsection{Activity Centric Process Models}

Activity centric process model express processes by adding constraints to activities.
Those constraints can be either data dependencies or control flow dependencies.
Data dependencies express data objects which will be read and written by the activity;
whereas control flow allows us to create an order between activities of one model.

The Table \ref{tbl:acpm_elements} shows the elements of Activity Centric Process Models the interface they implement and the restrictions they add to the default behavior.

\begin{table}[h]
	\centering
	\begin{tabular}{|l|l|p{8cm}|}
		\hline
		\textbf{class name} & \textbf{interface} & \textbf{restrictions}\\
		\hline
		Activity & INode & \begin{itemize}
					\item Only Control Flow and Data Flow edges are supported
					\item There must be only one incoming control flow edge
					\item There must be only one outgoing control flow edge
				\end{itemize}\\
		\hline
		ActivityCentricProcessModel & IModel & \begin{itemize}
			\item Node must be one of the supported types.
		\end{itemize}\\
		\hline
		ControlFlow & IEdge & \begin{itemize}
									\item Source must be either a Gateway, Event or Activity
								\end{itemize}\\
		\hline
		DataFlow & IEdge & \begin{itemize}
								\item The source must be either a Activity or DataObject
								\item If the source is an Activity the target must be a DataObject and vise versa
							\end{itemize}\\
		\hline
		DataObject & INode & \begin{itemize}
								\item All incoming and outgoing edges must be of type DataFlow
							\end{itemize}\\
		\hline
		Event & INode & \begin{itemize}
							\item An event can have only one edge
							\item Only an Start Event can have an outgoing edge
							\item Only an End Event can have an incoming edge
						\end{itemize}\\
		\hline
		Gateway & INode & \begin{itemize}
							\item Every Edge must be of type ControlFlow.
							\item An additional type attribute defines if it is an XOR or an AND
						\end{itemize}\\
		\hline
	\end{tabular}
	\caption{Elements used by the activity centric process Model}
	\label{tbl:acpm_elements}
\end{table}
\subsubsection{Object Life Cycle}

Object Life Cycles are state transitions systems for one data class.
They express all possible states and transitions between them.
They can used in various ways for example for conformance checking.

The structure of an Object life cycle is simple.
Every model has exactly one start node and should have at least one final node.
There is no limit for other nodes.
All these models will be connected using state transitions.
In BPM a state transition normally represents an action which alters the data object.
Table \ref{tbl:olc_elements} describes the possible elements used inside a object life cycle.

The Table \ref{tbl:acpm_elements} shows the elements of Activity Centric Process Models the interface they implement and the restrictions they add to the default behavior.

\begin{table}[h]
	\centering
	\begin{tabular}{|l|l|p{10cm}|}
		\hline
		\textbf{class name} & \textbf{interface} & \textbf{restrictions}\\
		\hline
		Activity & INode & \begin{itemize}
					\item Only Control Flow and Data Flow edges are supported
					\item There must be only one incoming control flow edge
					\item There must be only one outgoing control flow edge
				\end{itemize}\\
		\hline
		ObjectLifeCycle & IModel & \begin{itemize}
			\item Every node must be of type DataObjectState
			\item There should be exactly one initial state
			\item THere should be at least one final state
		\end{itemize}\\
		\hline
		DataObjectState & INode & \begin{itemize}
									\item All incoming and outgoing edges must be of type StateTransition
								\end{itemize}\\
		\hline
		StateTransition & IEdge & \begin{itemize}
									\item Source and target must be of type DataObjectState
								\end{itemize}\\
		\hline
	\end{tabular}
	\caption{Elements used by the object life cycle model}
	\label{tbl:olc_elements}
\end{table}

In addition there is another package \textit{synchronize}. This package contains the synchronized Object Life Cycle (Table \ref{tbl:synchronize_olc}).

\begin{table}[h]
	\centering
	\begin{tabular}{|l|l|p{8cm}|}
		\hline
		\textbf{class name} & \textbf{interface} & \textbf{restrictions}\\
		\hline
		SynchronizedObjectLifeCycle & IModel & \begin{itemize}
					\item Initialized from multiple ObjectLifeCycles
					\item Altering is not possible
					\item Additional Synchronization edges can be created
				\end{itemize}\\
		\hline
	\end{tabular}
	\caption{The SynchronizedObjectLifeCycle}
	\label{tbl:synchronize_olc}
\end{table}

\subsection{How to convert a Process Model}

Once you have created process models with the classes described above (or with any subclasses) you can use the elements of the \textit{converter} package to convert these models.
The \textit{subpackages} indicate the source of the generation.
\textit{OLC} subpackage transforms object life cycles, \textit{activity\_centric} Activity Centric Process Models and \textit{pcm} Production Case Management scenarios (A collection of Activity Centric Process Models).
Figure \ref{fig:transformations} shows which transformations can be made.
\begin{figure}[h]
\centering
\includegraphics{images/transformation.PNG}
\caption{Possible transformations}
\label{fig:transformations}
\end{figure}